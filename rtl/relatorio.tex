\documentclass[conference]{IEEEtran}
\IEEEoverridecommandlockouts
\usepackage{cite}
\usepackage{amsmath,amssymb,amsfonts}
\usepackage{algorithmic}
\usepackage{graphicx}
\usepackage{textcomp}
\usepackage{xcolor}
\usepackage{listings}
\usepackage{url}
\usepackage[utf8]{inputenc}
\usepackage[portuguese]{babel}
\usepackage{tikz}
\usetikzlibrary{arrows.meta,automata,positioning}

% Configuração para código VHDL
\lstdefinestyle{vhdl}{
    language=VHDL,
    basicstyle=\ttfamily\small,
    keywordstyle=\color{blue}\bfseries,
    commentstyle=\color{green!60!black},
    stringstyle=\color{red},
    numbers=left,
    numberstyle=\tiny\color{gray},
    frame=single,
    breaklines=true,
    captionpos=b
}

\def\BibTeX{{\rm B\kern-.05em{\sc i\kern-.025em b}\kern-.08em
    T\kern-.1667em\lower.7ex\hbox{E}\kern-.125emX}}

\begin{document}

\title{Implementação em Hardware de Sistema de Transposição de Matrizes Utilizando VHDL: Uma Abordagem Configurável para Aplicações de Processamento Digital de Sinais}

\author{
Marlon Vieira da Silva (20150154),\\
Vitor Luiz Paz Roseno da Silva (20202147),\\
Tiago Bucar Vasconcelos Nazari (22200377), \\
Victor Henrique Labes de Figueiredo (22200378),\\
Carlos Benedito Hayden de Albuquerque Junior (23100455), \\
Douglas Lessa Graciosa (24250313),\\
Josenei Montibeller (18250076),\\
Isis Rosa de Lima (22201733) \\
\textit{Departamento de Informática e Estatística} \\
\textit{Universidade Federal de Santa Catarina} \\
Florianópolis, Brasil 
\and
Prof. Ismael Seidel \\
\textit{Departamento de Informática e Estatística} \\
\textit{Universidade Federal de Santa Catarina} \\
Florianópolis, Brasil
}

\maketitle

\begin{abstract}
Este artigo apresenta uma implementação em hardware abrangente de um sistema de transposição de matrizes utilizando VHDL. O projeto apresenta uma arquitetura configurável capaz de manipular matrizes de até 255×255 elementos com largura de dados parametrizável. O sistema emprega um controlador de máquina de estados finitos coordenado com uma unidade de caminho de dados para gerenciar eficientemente os padrões de acesso à memória necessários para operações de transposição de matrizes. A implementação demonstra a aplicação prática de princípios de projeto digital no desenvolvimento de hardware especializado para operações matriciais comumente requeridas em aplicações de processamento digital de sinais.
\end{abstract}

\begin{IEEEkeywords}
VHDL, transposição de matrizes, máquina de estados finitos, processamento digital de sinais, projeto de hardware, FPGA
\end{IEEEkeywords}

\section{Introdução}

A transposição de matrizes é uma operação fundamental em álgebra linear e aplicações de processamento digital de sinais. A operação envolve a troca das linhas e colunas de uma matriz, onde o elemento $A_{ij}$ torna-se $A_{ji}$ na matriz transposta. Embora implementações em software sejam comuns, soluções baseadas em hardware oferecem vantagens significativas em termos de velocidade e eficiência energética para aplicações em tempo real.

Este trabalho apresenta uma implementação completa em VHDL de um sistema de transposição de matrizes projetado para implantação em FPGA. A arquitetura do sistema compreende um controlador comportamental implementado como uma máquina de estados finitos (FSM) e um caminho de dados estrutural que gerencia o endereçamento complexo necessário para transposição eficiente de matrizes.

As principais contribuições deste trabalho incluem: (1) uma arquitetura de hardware parametrizável suportando matrizes de até 255×255, (2) um padrão eficiente de acesso à memória que minimiza ciclos de processamento, (3) uma abordagem de projeto modular facilitando integração com sistemas maiores, e (4) verificação abrangente através de simulação comportamental.

\section{Principais Características}

\begin{itemize}
    \item Arquitetura modular e parametrizável.
    \item Separação clara entre controle (FSM) e datapath.
    \item Biblioteca de componentes reutilizáveis (registradores, multiplexadores, comparadores, somadores).
    \item Suporte a diferentes larguras de dados e tamanhos de matriz via generics.
    \item Facilidade de integração com memórias externas.
\end{itemize}

\section{Algoritmo Utilizado}

O sistema implementa um algoritmo sequencial para varredura e processamento de elementos de duas matrizes (ou blocos de dados), controlado por uma máquina de estados finitos (FSM). O fluxo básico é:

\begin{enumerate}
    \item Inicialização dos contadores e registradores com os parâmetros de tamanho da matriz.
    \item Laço de leitura dos elementos das matrizes A e B.
    \item Processamento dos dados lidos (armazenamento, comparação, etc.).
    \item Escrita dos resultados e atualização dos endereços.
    \item Repetição do processo até o término do bloco/matriz.
\end{enumerate}

\section{Arquitetura do Sistema}

O sistema é composto por dois blocos principais:

\begin{itemize}
    \item \textbf{Bloco de Controle (\texttt{matriz\_bc}):} Implementa a FSM responsável por coordenar as operações de leitura, escrita e controle dos contadores/endereço.
    \item \textbf{Bloco Operativo (\texttt{matriz\_bo}):} Realiza o processamento dos dados, gerenciamento dos contadores, geração de endereços e armazenamento temporário dos dados.
\end{itemize}

A integração entre os blocos é feita no módulo top-level (\texttt{matriz.vhdl}), que conecta os sinais de controle, dados e endereços entre os módulos internos e as interfaces externas.

\subsection{Interface de Top Level}

A interface do sistema acomoda dimensões de matriz configuráveis e largura de dados através de parâmetros genéricos. As portas primárias incluem:

\begin{itemize}
\item \texttt{clk}, \texttt{rst\_a}: Sinais padrão de clock e reset assíncrono
\item \texttt{enable}: Sinal de ativação do sistema
\item \texttt{num\_linhas}, \texttt{num\_colunas}: Entradas das dimensões da matriz
\item \texttt{elemento1\_o}, \texttt{elemento2\_o}: Entradas de dados de leitura da memória
\item \texttt{elemento1\_t}, \texttt{elemento2\_t}: Saídas de dados de escrita da memória
\item \texttt{address1}, \texttt{address2}: Endereços de memória dual-port
\item \texttt{read\_mem}, \texttt{write\_mem}: Sinais de controle da memória
\item \texttt{done}: Indicador de conclusão da operação
\end{itemize}

A interface de memória dual-port permite acesso simultâneo a duas localizações de memória, melhorando significativamente o throughput comparado a implementações single-port.

\subsection{Bloco de Controle}

O bloco de controle (\texttt{matriz\_bc}) implementa uma FSM de 8 estados que orquestra a operação de transposição. A sequência de estados segue um protocolo cuidadosamente projetado:

\begin{enumerate}
\item \texttt{S0}: Estado inativo, aguardando sinal de habilitação
\item \texttt{S1}: Inicializar contadores e registradores
\item \texttt{S2}: Calcular endereços de memória para posição atual
\item \texttt{S3}: Ler primeiro elemento da matriz
\item \texttt{S4}: Ler segundo elemento da matriz
\item \texttt{S5}: Escrever primeiro elemento na posição transposta
\item \texttt{S6}: Escrever segundo elemento na posição transposta
\item \texttt{S7}: Atualizar contadores e verificar conclusão
\end{enumerate}

\subsection{Implementação da Máquina de Estados Finitos}

A implementação da FSM usa tipos enumerados para representação de estados, garantindo código sintetizável:

\begin{lstlisting}[style=vhdl]
type tipo_estado is(
    S0, S1, S2, S3, S4, S5, S6, S7
);
signal Eatual, Eprox : tipo_estado;
\end{lstlisting}

\begin{figure}[htbp]
\centering
\resizebox{\columnwidth}{!}{%
\begin{tikzpicture}[->,>=Stealth,auto,node distance=2cm,semithick]
  \tikzstyle{every state}=[draw=black,thick,fill=gray!10,minimum size=5pt]

 % Estados posicionados conforme solicitado
  \node[state,initial] (S0) {S0};
  \node[state,right=5cm of S0] (S1) {S1};
  \node[state,below=1cm of S1] (S2) {S2};
  \node[state,below=1cm of S2] (S3) {S3};
  \node[state,below=1cm of S3] (S4) {S4};
  \node[state,left=3cm of S4] (S5) {S5};
  \node[state,left=3cm of S5] (S6) {S6};
  \node[state,left=3cm of S2] (S7) {S7};

  % Transições principais
  \path (S0) edge[bend left=10] node[above] {iniciar=1} (S1);
  \path (S1) edge node[right] {} (S2);
  \path (S2) edge node[right] {} (S3);
  \path (S3) edge node[right] {} (S4);
  \path (S4) edge node[below] {} (S5);
  \path (S5) edge node[below] {} (S6);

  % S6 para S2 (loop)
  \path (S6) edge[bend right=10] node[right] {comp\_p\_out=1} (S2);

  % S6 para S7
  \path (S6) edge[bend left=30] node[left] {comp\_p\_out=0} (S7);

  % S7 para S0
  \path (S7) edge[bend left=15] node[left] {igual=1} (S0);

  % S7 para S2
  \path (S7) edge[bend left=10] node[above] {igual=0} (S2);

  % S0 loop
  \path (S0) edge[loop above] node {iniciar=0} (S0);
\end{tikzpicture}
}
\caption{Diagrama de estados da FSM matriz\_bc}
\end{figure}


As transições de estado dependem de entradas de controle (\texttt{iniciar}, \texttt{comp\_p\_out}, \texttt{igual}) e seguem um padrão determinístico que garante conclusão da operação.


O controlador gera onze sinais de controle distintos que coordenam as operações do caminho de dados. Sinais críticos incluem habilitações de contador (\texttt{cn}, \texttt{cp}), cargas de registrador (\texttt{car\_A}, \texttt{car\_B}), e habilitações de cálculo de endereço (\texttt{cend1}, \texttt{cend2}).

\subsection{Bloco Operativo}

O bloco operativo (\texttt{matriz\_bo}) implementa os aspectos computacionais da transposição de matrizes. Componentes principais incluem:

\subsubsection{Gerenciamento de Contadores}
Dois contadores independentes rastreiam a posição atual da matriz:
\begin{itemize}
\item \texttt{n\_count}: Contador de linha (0 a N-1)
\item \texttt{p\_count}: Contador de coluna (0 a P-1)
\end{itemize}

As atualizações dos contadores seguem um padrão de loop aninhado, onde o contador de coluna incrementa para cada elemento, e o contador de linha incrementa quando uma linha completa é processada.

\subsubsection{Geração de Endereços}
O componente \texttt{element\_pos} calcula endereços de memória usando a fórmula:
\begin{align}
\text{address}_1 &= N \times \text{n\_count} + \text{p\_count} \\
\text{address}_2 &= N \times \text{p\_count} + \text{n\_count}
\end{align}

Este esquema de endereçamento garante que os elementos $A_{ij}$ e $A_{ji}$ sejam acessados simultaneamente, possibilitando transposição eficiente.

\subsubsection{Lógica de Comparação}
Dois comparadores determinam os limites da operação:
\begin{itemize}
\item \texttt{comparador\_n}: Detecta conclusão de linha (n\_count = N-1)
\item \texttt{comparador\_p}: Habilita incremento do contador de coluna (p\_count + 1 < P)
\end{itemize}

\section{Detalhes de Implementação}

\subsection{Estratégia de Parametrização}

O projeto emprega generics em VHDL para alcançar configurabilidade:
\begin{lstlisting}[style=vhdl]
generic(
    W : positive := 8;  -- Largura da dimensão da matriz
    tamanho : positive := 8  -- Largura do elemento de dados
);
\end{lstlisting}

\begin{itemize}
\item \textbf{Generics:}
\begin{itemize}
\item \texttt{W}: largura dos endereços (bits), padrão 8
\item \texttt{tamanho}: largura dos dados (bits), padrão 1
\end{itemize}
\end{itemize}

\subsection{Protocolo de Interface de Memória}

O sistema implementa uma interface de memória dual-port com controle separado de leitura/escrita. A sequência de protocolo para cada passo de transposição envolve:

\begin{enumerate}
\item Cálculo e registro de endereços
\item Operação de leitura da memória com apresentação de endereços
\item Captura de dados em registradores temporários
\item Operação de escrita da memória com apresentação de dados
\end{enumerate}

Este protocolo garante integridade dos dados e suporta vários requisitos de temporização de memória.

\section{Projeto da Biblioteca de Componentes}

A implementação aproveita uma biblioteca abrangente de componentes que inclui:

\subsection{Blocos de Construção Básicos}
\begin{itemize}
\item \texttt{unsigned\_register}: Registrador parametrizável com habilitação
\item \texttt{unsigned\_adder}: Adição sem sinal de N bits
\item \texttt{mux\_2to1}, \texttt{mux\_3to1}: Componentes multiplexadores
\end{itemize}

\subsection{Componentes Especializados}
\begin{itemize}
\item \texttt{comparador\_n}, \texttt{comparador\_p}: Detecção de fronteira
\item \texttt{element\_pos}: Unidade de cálculo de endereços
\end{itemize}

Esta abordagem modular facilita reutilização de projeto e simplifica manutenção.

\section{Aplicações e Extensões}

O sistema de transposição de matrizes encontra aplicações em:

\begin{itemize}
\item Algoritmos de processamento digital de sinais
\item Operações de processamento de imagens
\item Computações de álgebra linear
\item Reorganização de dados em aplicações de streaming
\end{itemize}

Extensões potenciais incluem:
\begin{itemize}
\item Transposição baseada em blocos para matrizes maiores
\item Implementação pipeline para aumento de throughput
\item Integração com unidades de multiplicação de matrizes
\item Suporte para matrizes de números complexos
\end{itemize}

\section{Conclusão}

Este artigo apresentou uma implementação abrangente em VHDL de um sistema de transposição de matrizes apresentando dimensões configuráveis e utilização eficiente de memória. O projeto demonstra a aplicação efetiva de princípios de máquina de estados finitos e metodologias de projeto estrutural na criação de hardware especializado para operações matriciais.

A arquitetura modular e projeto parametrizável permitem adaptação a vários requisitos de aplicação mantendo eficiência de hardware. A interface de memória dual-port e esquema de endereçamento otimizado fornecem vantagens significativas de desempenho sobre implementações em software.

\section*{Agradecimentos}

Os autores agradecem ao Laboratório de Sistemas Digitais da UFSC por fornecer o ambiente de desenvolvimento e infraestrutura de teste que tornou este trabalho possível.

\begin{thebibliography}{00}
\bibitem{b1} **Material didático da disciplina INE5406 - Sistemas Digitais (UFSC)**, incluindo slides e exemplos fornecidos em aula. 
\bibitem{b2} VAHID, Frank. Sistemas Digitais: projeto, otimização e HDLs. Porto Alegre: Bookman, 2008. ISBN978-85-7780-190-9
\bibitem{b3} PATTERSON, David A.; HENNESSY, John L. Organização e Projeto de Computadores: a interface
hardware/software. Rio de Janeiro: Elsevier, 2005. ISBN 85-352-1521-2
\bibitem{b4} IEEE Computer Society, "IEEE Standard VHDL Language Reference Manual," IEEE Std 1076-2008, 2009.

\end{thebibliography}

\end{document}